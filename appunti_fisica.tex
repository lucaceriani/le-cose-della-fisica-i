\documentclass[12pt, oneside]{book}
\usepackage[italian]{babel}
\usepackage{amsmath}
\usepackage[a4paper]{geometry}
\linespread{1.33}
\usepackage[utf8]{inputenc}
\usepackage{enumerate}
%\usepackage{afterpage}
\usepackage{makeidx}

\newcommand{\pvec}[1]{\vec{#1}\mkern2mu\vphantom{#1}}

\title{\vspace*{\fill}\Huge{\textbf{Le cose della Fisica I}}} % Title
\author{Luca Ceriani\bigskip}
\date{Torino, \today\vspace*{\fill}}

\begin{document}
\maketitle
\newpage
%\afterpage{\null\thispagestyle{empty}\newpage}
\cleardoublepage

\chapter{Cinematica}

\section{Velocità}
Tenendo conto della definizione di derivata di un vettore, la velocità si può scrivere come
\[\vec{v}(t)=\frac{d\vec{r}(t)}{dt}\]
Si consideri poi il vettore tangente alla traiettoria $\hat{u}_t$, esso può essere espresso come
\[\hat{u}_t=\lim_{\Delta s\to 0}\dfrac{\Delta\vec{r}}{\Delta s}
= \frac{d\vec{r}}{ds}\]
e quindi
\[\vec{v}=\frac{d\vec{r}}{dt}=\frac{d\vec{r}}{ds}\frac{ds}{dt}=\frac{ds}{dt}\hat{u}_t~.\]
Notare che l'espressione $\frac{ds}{dt}$ prende il nome di velocità scalare:
\[v_s=\frac{ds}{dt}~.\]

\section{Accelerazione}
Conoscendo il vettore velocità $\vec{v}(t)$, si considera accelerazione la variazione di questo vettore nel tempo. In formule:
\[\vec{a}(t)=\frac{d\vec{v}(t)}{dt}=\frac{d^2 \vec{r}(t)}{dt^2}~.\]
Se ci si riferisce ad una traiettoria $\vec{s}(t)$, utilizzando le relazioni trovate in precedenza si avrà
\[\vec{a}(t)=\frac{d\vec{v}(t)}{dt}=\frac{d}{dt}(v_s \hat{u}_t)=\frac{dv_s}{dt}\hat{u}_t+v_s\frac{d\hat{u}_t}{dt}\]

\section{Classificazione dei moti elementari}
Fissando l'attenzione sull'\textit{equazione oraria}, possiamo definire due importanti calssi di moto:
\begin{enumerate}
	\item moti uniformi ($v_s\equiv \dot{s} = cost = \dot{s}_0$);
	\item{moti con accelerazione tangenziale costante ($a_t\equiv \ddot{s} = cost = \ddot{s}_0$).}
\end{enumerate}
Dal punto di vista geometrico, i moti con traiettoria notevole sono:
\begin{enumerate}
	\item i \emph{moti rettilinei}, caratterizzati da $\rho \rightarrow \infty$;
	\item i \emph{moti circolari}, in cui $\rho = cost$.
\end{enumerate}
Dove $\rho$ è il raggio di curvatura.

\section{Moti rettilinei}
\begin{center}
\begin{tabular}{r l}
\textbf{Moto rettilineo uniforme} & $x(t)=v_0 t+x_0$\\[12pt]
\textbf{Moto rettilineo uniformemente accelerato} & $x(t)=\frac{1}{2}a_0 t^2 + v_0 t +x_0$\\
& $\dot{x}(t)=v(t)=a_0 t + v_0$\\
\end{tabular}
\end{center}

\section{Moti circolari}
Per i moti circolari di definiscono alcune grandezze. Per un generico moto circolare si può scrivere:
\[\vec{v}(t)=\vec{\omega}(t)\times\vec{r}(t)~.\]
dove
\[\vec{\omega}(t)=\frac{d\theta (t)}{dt}\hat{k}=\dot{\theta}\hat{k}~.\]
Si può anche definire un'\textbf{accelerazione angolare} come segue:
\[\vec{\alpha}(t)=\frac{d\vec{\omega}}{dt}=\frac{d\omega}{dt}\hat{k}=\ddot{\theta}\hat{k}~,\]
e quindi
\begin{equation}
\label{ainaplha}
\vec{a}=\vec{\alpha}\times\vec{r}+\vec{\omega}\times(\vec{\omega}\times\vec{r})~.
\end{equation}
\bigskip

\section{Moto circolare uniforme}
Detto $T$ il periodo del moto, si definisce la \textbf{pulsazione} come:
\[\omega_0=2\pi\nu=\dfrac{2\pi}{T}~.\]

\section{Moto oscillatorio armonico}
Si consideri un corpo che si muove lungo una retta con la seguente legge oraria:
\[x(t)=A\cos(\omega_0 t+\varphi_0)~,\]
possiamo calcolare la velocità e l'accelerazione per ogni instante di $t$:
\begin{align*}
	\begin{cases}
		v_x=-\omega_0 A \sin(\omega_0 t + \varphi_0)\\
		a_x=-\omega_0^2 A \cos(\omega_0 t + \varphi_0)=-\omega_0^2x
	\end{cases}
\end{align*}
L'equazione differenziale del moto oscillatorio armonico, su una traiettoria qualunque vale:
\[\ddot{s}+\omega_0^2s=0\]

\section{Moto parabolico}
In breve le formule del moto parabolico nelle componenti $x$ e $y$:
\begin{align*}
	\begin{cases}
		x(t)=(v_0 \cos(\alpha)t)\\
		y(t)=(v_0 \sin(\alpha)t-\frac{1}{2}gt^2)
	\end{cases}~,
\end{align*}
mentre la gittata, intesa come la distanza percorsa su $x$ se il lancio è avvenuto ad altezza 0 sarà:
\[y=\dfrac{v_{0y}}{v_{0x}}x-\dfrac{1}{2}\dfrac{g}{v_{0x}^2}x^2~.\]

\section{Velocità e accelerazione in coordinate polari piane}
Considerando un sistema di coordinate polari, nei quali la posizione è definita da $\vec r = r \hat u_r$, si definiscono velocità e accelerazione come segue:
\begin{center}\begin{tabular}{r c l}
$\vec v$ & $=$ & $\dot r \hat u_r + r\dot\theta \hat u_\theta$\\
$\vec a$ & $=$ &$(\ddot r - r\dot \theta^2)\hat u_r + (2\dot r \dot \theta + r\ddot \theta)\hat u_\theta$
\end{tabular}\end{center}

\section{Problema inverso della cinematica}
Dato che $\vec v = \dfrac{d\vec r}{dt}$ e $\vec a = \dfrac{d\vec{v}}{dt}$, si possono scrivere le relazioni inverse in termini di integrali, ovvero:
\begin{center}\begin{tabular}{r c l}
$\vec r (t)$ & $=$ & $\vec r (t_0) + {\displaystyle \int_{t_0}^t \vec v (t')dt'}$\\[12pt]
$\vec v (t)$ & $=$ & $\vec v (t_0) + {\displaystyle \int_{t_0}^t \vec a (t')dt'}$
\end{tabular}\end{center}

\section{Trasformazione di velocità e di accelerazione}
Se si suppongo lo \emph{spazio e il tempo assoluti} e considerato $\vec{R}=\overrightarrow{OO'}$ e $\pvec{r}'=\vec{r}+\vec{R}$, allora si domostra facilmente che:
\[\vec{v}=\pvec{v}'+\vec{v_\tau}\]
dove $\vec{v_\tau}$ è detta \emph{velocità di trascinamento}.
Posto $\vec{V}(t)=\dfrac{d\vec{R}}{dt}$, allore la velocità di trascinamento si potrà esprimere nella forma
\begin{equation}
\label{vtau}
\vec{v}_\tau(t)=\vec{V}(t)+\vec{\omega}(t)\times[\vec{r}(t)-\vec{R}(t)]~.
\end{equation}
Questo perché la velocità di trascinamento non è uguale per ogni punto di $S'$ (qualsiasi caso di moto non puramente traslatorio). In particolare si possono dietinguere due casi:
\begin{enumerate}[a)]
	\item $\vec{\omega}=0$\\
	in questo caso si parla di \textbf{moto traslatorio}, tutti i punti di $S'$ si muovono alla stessa velocità: $\vec{v}_\tau(t)=\vec{V}$~;
	\item $\vec{\omega}\neq 0$\\
	$\vec{v}_\tau$ varia da punto a punto, se si considera la sola coponente di rotazione la (\ref{vtau}) può essere riscritta come
	\[\vec{v}_\tau=\vec{\omega}\times(\vec{r}-\vec{R})~.\]
\end{enumerate}
Si dimostra che la dipendenza tra $\vec{a}$ e $\pvec{a}'$ è più conmplessa e risulta essere
\begin{equation}
\label{aaprime}
\vec{a}=\pvec{a}' + \vec{a}_\tau + \vec{a}_{co}~.
\end{equation}
\newpage
Dove, considerando la (\ref{ainaplha}) e la (\ref{aaprime}) si può scrivere:
\[\vec a_\tau=\vec A+\vec \alpha \times(\vec{r}-\vec{R})+\vec{\omega}\times[\vec{\omega}\times (\vec{r}-\vec{R})]\]
e
\[\vec{a}_{co}=2\vec{\omega}\times\pvec v'\]
\section{Trasformazioni di Galileo}
In breve, tenendo conto che $\vec R = \vec V t$, allora si dicono trasformazioni di Galielo le seguenti
\begin{align*}
	\begin{cases}
		\vec r = \pvec r' + \vec V t'\\
		t=t'
	\end{cases}~,
\end{align*}
e per $\vec V = costante$ si ha:
\begin{align*}
	\begin{cases}
		\vec v = \pvec v' + \vec V\\
		\vec a = \pvec a'
	\end{cases}~.
\end{align*}
\chapter{Dinamica}
\end{document}








\documentclass[12pt, oneside]{book}
\usepackage[italian]{babel}
\usepackage{amsmath}
\usepackage[a4paper]{geometry}
\linespread{1.3}
\usepackage[utf8]{inputenc}
\usepackage{enumerate}
%\usepackage{afterpage} %decommentare per agina vuota dopo il titolo
\usepackage{makeidx}

\usepackage{endnotes} %per le note a fine libro
\renewcommand{\notesname}{Note}

\newcommand{\pvec}[1]{\vec{#1}\mkern2mu\vphantom{#1}}

\title{\vspace*{\fill}\Huge{\textbf{Le cose della Fisica I}}}
\author{Luca Ceriani\bigskip}
\date{Torino, \today\vspace*{\fill}}

\begin{document}
\frenchspacing
\maketitle
\newpage
%\afterpage{\null\thispagestyle{empty}\newpage} % decommentare per pagina vuota dopo il titolo
\cleardoublepage
\tableofcontents


%======================================
%				CINEMATICA
%======================================


\chapter{Cinematica}

\section{Velocità}
Tenendo conto della definizione di derivata di un vettore, la velocità si può scrivere come
\[\vec{v}(t)=\frac{d\vec{r}(t)}{dt}\]
Si consideri poi il vettore tangente alla traiettoria $\hat{u}_t$, esso può essere espresso come
\[\hat{u}_t=\lim_{\Delta s\to 0}\dfrac{\Delta\vec{r}}{\Delta s}
= \frac{d\vec{r}}{ds}\]
e quindi
\[\vec{v}=\frac{d\vec{r}}{dt}=\frac{d\vec{r}}{ds}\frac{ds}{dt}=\frac{ds}{dt}\hat{u}_t~.\]
Notare che l'espressione $\frac{ds}{dt}$ prende il nome di velocità scalare:
\[v_s=\frac{ds}{dt}~.\]

\section{Accelerazione}
Conoscendo il vettore velocità $\vec{v}(t)$, si considera accelerazione la variazione di questo vettore nel tempo. In formule:
\[\vec{a}(t)=\frac{d\vec{v}(t)}{dt}=\frac{d^2 \vec{r}(t)}{dt^2}~.\]
Se ci si riferisce ad una traiettoria $\vec{s}(t)$, utilizzando le relazioni trovate in precedenza si avrà
\[\vec{a}(t)=\frac{d\vec{v}(t)}{dt}=\frac{d}{dt}(v_s \hat{u}_t)=\frac{dv_s}{dt}\hat{u}_t+v_s\frac{d\hat{u}_t}{dt}\]

\section{Classificazione dei moti elementari}
Fissando l'attenzione sull'\textit{equazione oraria}, possiamo definire due importanti calssi di moto:
\begin{enumerate}
	\item moti uniformi ($v_s\equiv \dot{s} = cost = \dot{s}_0$);
	\item{moti con accelerazione tangenziale costante ($a_t\equiv \ddot{s} = cost = \ddot{s}_0$).}
\end{enumerate}
Dal punto di vista geometrico, i moti con traiettoria notevole sono:
\begin{enumerate}
	\item i \emph{moti rettilinei}, caratterizzati da $\rho \rightarrow \infty$;
	\item i \emph{moti circolari}, in cui $\rho = cost$.
\end{enumerate}
Dove $\rho$ è il raggio di curvatura.

\section{Moti rettilinei}
\begin{center}
\begin{tabular}{r l}
\textbf{Moto rettilineo uniforme} & $x(t)=v_0 t+x_0$\\[12pt]
\textbf{Moto rettilineo uniformemente accelerato} & $x(t)=\frac{1}{2}a_0 t^2 + v_0 t +x_0$\\
& $\dot{x}(t)=v(t)=a_0 t + v_0$\\
\end{tabular}
\end{center}

\section{Moti circolari}
Per i moti circolari di definiscono alcune grandezze. Per un generico moto circolare si può scrivere:
\[\vec{v}(t)=\vec{\omega}(t)\times\vec{r}(t)~.\]
dove
\[\vec{\omega}(t)=\frac{d\theta (t)}{dt}\hat{k}=\dot{\theta}\hat{k}~.\]
Si può anche definire un'\textbf{accelerazione angolare} come segue:
\[\vec{\alpha}(t)=\frac{d\vec{\omega}}{dt}=\frac{d\omega}{dt}\hat{k}=\ddot{\theta}\hat{k}~,\]
e quindi
\begin{equation}
\label{ainaplha}
\vec{a}=\vec{\alpha}\times\vec{r}+\vec{\omega}\times(\vec{\omega}\times\vec{r})~.
\end{equation}
\bigskip

\section{Moto circolare uniforme}
Detto $T$ il periodo del moto, si definisce la \textbf{pulsazione} come:
\[\omega_0=2\pi\nu=\dfrac{2\pi}{T}~.\]

\section{Moto oscillatorio armonico}
Si consideri un corpo che si muove lungo una retta con la seguente legge oraria:
\[x(t)=A\cos(\omega_0 t+\varphi_0)~,\]
possiamo calcolare la velocità e l'accelerazione per ogni instante di $t$:
\begin{align*}
	\begin{cases}
		v_x=-\omega_0 A \sin(\omega_0 t + \varphi_0)\\
		a_x=-\omega_0^2 A \cos(\omega_0 t + \varphi_0)=-\omega_0^2x
	\end{cases}
\end{align*}
L'equazione differenziale del moto oscillatorio armonico, su una traiettoria qualunque vale:
\begin{equation}
\label{diff oscillatorio}
\ddot{s}+\omega_0^2s=0
\end{equation}


\section{Moto parabolico}
In breve le formule del moto parabolico nelle componenti $x$ e $y$:
\begin{align*}
	\begin{cases}
		x(t)=(v_0 \cos(\alpha)t)\\
		y(t)=(v_0 \sin(\alpha)t-\frac{1}{2}gt^2)
	\end{cases}~,
\end{align*}
mentre la gittata, intesa come la distanza percorsa su $x$ se il lancio è avvenuto ad altezza 0 sarà:
\[y=\dfrac{v_{0y}}{v_{0x}}x-\dfrac{1}{2}\dfrac{g}{v_{0x}^2}x^2~.\]

\section{Velocità e accelerazione in coordinate polari piane}
Considerando un sistema di coordinate polari, nei quali la posizione è definita da $\vec r = r \hat u_r$, si definiscono velocità e accelerazione come segue:
\begin{center}\begin{tabular}{r c l}
$\vec v$ & $=$ & $\dot r \hat u_r + r\dot\theta \hat u_\theta$\\
$\vec a$ & $=$ &$(\ddot r - r\dot \theta^2)\hat u_r + (2\dot r \dot \theta + r\ddot \theta)\hat u_\theta$
\end{tabular}\end{center}

\section{Problema inverso della cinematica}
Dato che $\vec v = \dfrac{d\vec r}{dt}$ e $\vec a = \dfrac{d\vec{v}}{dt}$, si possono scrivere le relazioni inverse in termini di integrali, ovvero:
\begin{center}\begin{tabular}{r c l}
$\vec r (t)$ & $=$ & $\vec r (t_0) + {\displaystyle \int_{t_0}^t \vec v (t')dt'}$\\[12pt]
$\vec v (t)$ & $=$ & $\vec v (t_0) + {\displaystyle \int_{t_0}^t \vec a (t')dt'}$
\end{tabular}\end{center}

\section{Trasformazione di velocità e di accelerazione}
Se si suppongo lo \emph{spazio e il tempo assoluti} e considerato $\vec{R}=\overrightarrow{OO'}$ e $\pvec{r}'=\vec{r}+\vec{R}$, allora si domostra facilmente che:
\[\vec{v}=\pvec{v}'+\vec{v_\tau}\]
dove $\vec{v_\tau}$ è detta \emph{velocità di trascinamento}.
Posto $\vec{V}(t)=\dfrac{d\vec{R}}{dt}$, allore la velocità di trascinamento si potrà esprimere nella forma
\begin{equation}
\label{vtau}
\vec{v}_\tau(t)=\vec{V}(t)+\vec{\omega}(t)\times[\vec{r}(t)-\vec{R}(t)]~.
\end{equation}
Questo perché la velocità di trascinamento non è uguale per ogni punto di $S'$ (qualsiasi caso di moto non puramente traslatorio). In particolare si possono dietinguere due casi:
\begin{enumerate}[a)]
	\item $\vec{\omega}=0$\\
	in questo caso si parla di \textbf{moto traslatorio}, tutti i punti di $S'$ si muovono alla stessa velocità: $\vec{v}_\tau(t)=\vec{V}$~;
	\item $\vec{\omega}\neq 0$\\
	$\vec{v}_\tau$ varia da punto a punto, se si considera la sola coponente di rotazione la (\ref{vtau}) può essere riscritta come
	\[\vec{v}_\tau=\vec{\omega}\times(\vec{r}-\vec{R})~.\]
\end{enumerate}
Si dimostra che la dipendenza tra $\vec{a}$ e $\pvec{a}'$ è più conmplessa e risulta essere
\begin{equation}
\label{aaprime}
\vec{a}=\pvec{a}' + \vec{a}_\tau + \vec{a}_{co}~.
\end{equation}
Dove, considerando la (\ref{ainaplha}) e la (\ref{aaprime}) si può scrivere:
\[\vec a_\tau=\vec A+\vec \alpha \times(\vec{r}-\vec{R})+\vec{\omega}\times[\vec{\omega}\times (\vec{r}-\vec{R})]\]
e
\[\vec{a}_{co}=2\vec{\omega}\times\pvec v'\]
\section{Trasformazioni di Galileo}
In breve, tenendo conto che $\vec R = \vec V t$, allora si dicono trasformazioni di Galielo le seguenti
\begin{align*}
	\begin{cases}
		\vec r = \pvec r' + \vec V t'\\
		t=t'
	\end{cases}~,
\end{align*}
e per $\vec V = costante$ si ha:
\begin{align*}
	\begin{cases}
		\vec v = \pvec v' + \vec V\\
		\vec a = \pvec a'
	\end{cases}~.
\end{align*}


%======================================
%				DINAMICA
%======================================


\chapter{Dinamica}
\section{Le tre leggi di Newton}
Il problema tipico della meccanica è determinare il moto di un corpo che interagisce con l'ambiente circostante caratterizzato dalla presenza di altri corpi. Newton formula le tre leggi della dinamica che recitano:
\begin{enumerate}
\item \textit{``Ciascun corpo persisterà nel suo stato di quiete o di moto rettilineo uniforme finché non è indotto a cambiare il suo stato da forze che agiscono su di esso.''}
\item \textit{``L'accelerazione prodotta da una forza è proporzionale alla massa inerziale del corpo''}:
\[\vec F = m\vec a~.\]
\item \textit{``Ad ogni forza ne corrisponde una uguale e contraria.''}
\end{enumerate}
\section{Teorema delle quantità di moto}
Facendo riferimento alla seconda legge di Newton, si può affermare che ogni volta che un corpo cambia la sua quantità di moto è a causa di una forza e si può scrivere la seguente relazione
\endnote{$\dfrac{d\vec{p}}{dt}=\dfrac{d(m\vec{v})}{dt}=m\dfrac{d\vec{v}}{dt}=m\vec{a}=\vec{F}$}
\begin{equation}
\label{f=dp/dt}
\vec{F}=\dfrac{d\vec{p}}{dt}~.
\end{equation}

Si definisce poi l'impulso di una forza come l'azione di quella forza nel tempo, in formule:
\begin{equation}
\label{impulso}
\vec{J}=\int_{t1}^{t2}\vec{F}(t)dt
\end{equation}
Data l'equazione (\ref{f=dp/dt}) si può scrivere $\vec{F}dt=d\vec{p}$, da cui, considerando la (\ref{impulso}) si ricava che
\[\vec J =\int_{t1}^{t2} d\vec{q}=\vec{q}_2-\vec{q}_1\equiv\Delta\vec{q}~.\]

\section{Legge di Hooke}
Nel caso di una molla ideale la forza prodotta dalla molla segue la \textit{Legge di Hooke} che recita come segue:
\[\vec{F}_e=-kx\hat{u}_x~.\]
Partendo da questa legge è facile dimostrare che un corpo di massa $m$ su cui agisce una forza elastica si muove di moto oscillatorio armonico, infatti
\[\vec{F}_e=-m\vec a \Rightarrow -kx=m\ddot x\Rightarrow \ddot x + \dfrac{k}{m}x=0~,\]
che è esattamente l'equazione differenziale (\ref{diff oscillatorio}), la cui soluzione generale è del tipo
\[x(t)=A\cos (\omega_0 t + \varphi_0) \textnormal{, con } \omega_0 = \sqrt{\dfrac{k}{m}}~.\]

\section{Pendolo semplice}
Le uniche forze che agiscono su di un pendolo semplice sono la forza di gravità e la reazione vincolare del filo. Per determinare l'equazione del moto, si può quindi scrivere
\[\vec F_g + \vec R = m\vec a~.\]
Utilizzando le coordinate intrinseche per descrivere la posizione del pendolo e tenuto conto dell'accelerazione centripeta, l'equazione del moto diventa:
\begin{align*}
\begin{cases}
-mg\sin (\theta)=m\ddot s\\
-mg\cos (\theta)+R=m\dfrac{\dot s^2}{L}
\end{cases}
\end{align*}
Ricordandosi che $\theta = \dfrac{s}{L}$, la prima equazione diventa:
\[\ddot s + g\sin\left(\dfrac{s}{L}\right)=0~,\]
che, per oscillazioni piccole si può riscrivere come
\[\ddot s +\dfrac{g}{L}s=0\]
che è la nota equazione del moto oscillatorio armonico di pulsazione $\omega_0 = \sqrt{\dfrac{g}{L}}$. L'oscillazione ha periodo
\[T=\dfrac{2\pi}{\omega_0}=2\pi\sqrt{\dfrac{L}{g}}~.\]

\section{Attrito radente statico}
Se si definisce con $\vec R_t$ la forza d'attrito che si oppone la moto, il valore massimo dell'attrito può essere scritto come $R_t^{max}=\mu_s N$ e la legge dell'attrito risulta
\[ R_t \leq \mu_s N \]

\section{Attrito randente dinamico}
Come l'attrito statico, l'attrito dinamico si presenta nella forma $R_t=\mu_d R_n$. La forza $\vec R_t$ è doretta in verso opposto al versore $\hat u_v$ della velocità relativa $v$ tra le due superfici di contatto, per cui:
\[\vec R_t = -\mu_dN\hat u_v\]


%======================================
%			ENERGIA E LAVORO
%======================================


\chapter{Energia e lavoro}
\section{Lavoro di una forza}
Per una forza$\vec F$ che non varia da punto a punto si definisce il lavoro di quella forza come prodotto tra la forza e lo spostamento:
\[W_{A\to B}=\vec{F}\cdot\vec{AB}\equiv\vec{F}\cdot\Delta\vec r~,\]
che, in modulo, diventa
\[W=F\Delta r\cos (\theta)~.\]
Si definisce lavoro elementare di una forza $dL=\vec{F}\cdot d\vec r$, integrando si ottiene:
\[L_{A\to B}=\int^B_A \vec{F}\cdot d\vec r~.\]

\section{Energia cinetica}
Si definisce l'energia cinetica associata a un corpo di massa $m$ e di velocità $v$ come
\[E_k=\dfrac{1}{2}mv^2~.\]
Si può dimostrare facilmente che
\begin{equation}
\label{lavoro = energia cinetica}
L_{A\to B}=E_k^A-E_k^B
\end{equation}
scrivendo
\[dL=\vec{F}\cdot d\vec{r}=m\vec{a}\cdot d\vec{r}=m\vec v \cdot\vec a dt\]
e poiché l'energia cinetica $E_k$ può essere scritta come $\frac{1}{2}m\vec v\cdot \vec v$, risulta
\[dE_k=d\left(\dfrac{1}{2}m\vec v \cdot\vec v\right) =m\vec v \cdot \vec a dt~,\]
quindi
\[dL=dE_k\]
che, interegrando diventa la (\ref{lavoro = energia cinetica}).

\section{Forze conservative}
Una forza si dice conservativa se e solo se il lavoro che essa compie su un oggetto generico dipende solo dalla posizione iniziale e quella finale indipendentemente dalla traiettoria seguita.\\
\`{E} ovvio che quindi che lungo un camminco chiuso, nel quale la posizione di partenza $P_1$ e la posizione di arrivo $P_2$ coincidano il lavoro svolto è nullo, questa relazione si puù scrivere come:
\[\oint_{P_1}^{P_2} \vec F \cdot d\vec r \equiv 0 ~,\]
questa relazione prende il nome di \emph{circuitazione}

\section{Energia potenziale}
Se una forza è conservativa, possiamo associare ad esse una quantità $E_p$ che prende il nome di \emph{energia potenziale}, che dipende soltanto dalla posizione della particella, in modo tale che valga:
\[W_{(P_1,P_2)}=E_p (P_1)-E_p (P_2)~.\]
Nel caso di una forza conservativa, la relazione precedente può essere scritta come 
\[W_{(P_1,P_2)}=E_p (P_1)-E_p (P_2)=E_{k,2}-E_{k,1}~.\]
Se si definisce l'\emph{energia meccanica totale} come $E\equiv E_k+E_p$, essa si conserva durante il moto della particella
\[E\equiv E_p + E_k=E_p+\dfrac{1}{2}mv^2=costante~.\]















































\newpage
\theendnotes
\end{document}
